% Options for packages loaded elsewhere
\PassOptionsToPackage{unicode}{hyperref}
\PassOptionsToPackage{hyphens}{url}
%
\documentclass[
]{article}
\usepackage{amsmath,amssymb}
\usepackage{lmodern}
\usepackage{iftex}
\ifPDFTeX
  \usepackage[T1]{fontenc}
  \usepackage[utf8]{inputenc}
  \usepackage{textcomp} % provide euro and other symbols
\else % if luatex or xetex
  \usepackage{unicode-math}
  \defaultfontfeatures{Scale=MatchLowercase}
  \defaultfontfeatures[\rmfamily]{Ligatures=TeX,Scale=1}
\fi
% Use upquote if available, for straight quotes in verbatim environments
\IfFileExists{upquote.sty}{\usepackage{upquote}}{}
\IfFileExists{microtype.sty}{% use microtype if available
  \usepackage[]{microtype}
  \UseMicrotypeSet[protrusion]{basicmath} % disable protrusion for tt fonts
}{}
\makeatletter
\@ifundefined{KOMAClassName}{% if non-KOMA class
  \IfFileExists{parskip.sty}{%
    \usepackage{parskip}
  }{% else
    \setlength{\parindent}{0pt}
    \setlength{\parskip}{6pt plus 2pt minus 1pt}}
}{% if KOMA class
  \KOMAoptions{parskip=half}}
\makeatother
\usepackage{xcolor}
\usepackage[margin=1in]{geometry}
\usepackage{color}
\usepackage{fancyvrb}
\newcommand{\VerbBar}{|}
\newcommand{\VERB}{\Verb[commandchars=\\\{\}]}
\DefineVerbatimEnvironment{Highlighting}{Verbatim}{commandchars=\\\{\}}
% Add ',fontsize=\small' for more characters per line
\usepackage{framed}
\definecolor{shadecolor}{RGB}{248,248,248}
\newenvironment{Shaded}{\begin{snugshade}}{\end{snugshade}}
\newcommand{\AlertTok}[1]{\textcolor[rgb]{0.94,0.16,0.16}{#1}}
\newcommand{\AnnotationTok}[1]{\textcolor[rgb]{0.56,0.35,0.01}{\textbf{\textit{#1}}}}
\newcommand{\AttributeTok}[1]{\textcolor[rgb]{0.77,0.63,0.00}{#1}}
\newcommand{\BaseNTok}[1]{\textcolor[rgb]{0.00,0.00,0.81}{#1}}
\newcommand{\BuiltInTok}[1]{#1}
\newcommand{\CharTok}[1]{\textcolor[rgb]{0.31,0.60,0.02}{#1}}
\newcommand{\CommentTok}[1]{\textcolor[rgb]{0.56,0.35,0.01}{\textit{#1}}}
\newcommand{\CommentVarTok}[1]{\textcolor[rgb]{0.56,0.35,0.01}{\textbf{\textit{#1}}}}
\newcommand{\ConstantTok}[1]{\textcolor[rgb]{0.00,0.00,0.00}{#1}}
\newcommand{\ControlFlowTok}[1]{\textcolor[rgb]{0.13,0.29,0.53}{\textbf{#1}}}
\newcommand{\DataTypeTok}[1]{\textcolor[rgb]{0.13,0.29,0.53}{#1}}
\newcommand{\DecValTok}[1]{\textcolor[rgb]{0.00,0.00,0.81}{#1}}
\newcommand{\DocumentationTok}[1]{\textcolor[rgb]{0.56,0.35,0.01}{\textbf{\textit{#1}}}}
\newcommand{\ErrorTok}[1]{\textcolor[rgb]{0.64,0.00,0.00}{\textbf{#1}}}
\newcommand{\ExtensionTok}[1]{#1}
\newcommand{\FloatTok}[1]{\textcolor[rgb]{0.00,0.00,0.81}{#1}}
\newcommand{\FunctionTok}[1]{\textcolor[rgb]{0.00,0.00,0.00}{#1}}
\newcommand{\ImportTok}[1]{#1}
\newcommand{\InformationTok}[1]{\textcolor[rgb]{0.56,0.35,0.01}{\textbf{\textit{#1}}}}
\newcommand{\KeywordTok}[1]{\textcolor[rgb]{0.13,0.29,0.53}{\textbf{#1}}}
\newcommand{\NormalTok}[1]{#1}
\newcommand{\OperatorTok}[1]{\textcolor[rgb]{0.81,0.36,0.00}{\textbf{#1}}}
\newcommand{\OtherTok}[1]{\textcolor[rgb]{0.56,0.35,0.01}{#1}}
\newcommand{\PreprocessorTok}[1]{\textcolor[rgb]{0.56,0.35,0.01}{\textit{#1}}}
\newcommand{\RegionMarkerTok}[1]{#1}
\newcommand{\SpecialCharTok}[1]{\textcolor[rgb]{0.00,0.00,0.00}{#1}}
\newcommand{\SpecialStringTok}[1]{\textcolor[rgb]{0.31,0.60,0.02}{#1}}
\newcommand{\StringTok}[1]{\textcolor[rgb]{0.31,0.60,0.02}{#1}}
\newcommand{\VariableTok}[1]{\textcolor[rgb]{0.00,0.00,0.00}{#1}}
\newcommand{\VerbatimStringTok}[1]{\textcolor[rgb]{0.31,0.60,0.02}{#1}}
\newcommand{\WarningTok}[1]{\textcolor[rgb]{0.56,0.35,0.01}{\textbf{\textit{#1}}}}
\usepackage{graphicx}
\makeatletter
\def\maxwidth{\ifdim\Gin@nat@width>\linewidth\linewidth\else\Gin@nat@width\fi}
\def\maxheight{\ifdim\Gin@nat@height>\textheight\textheight\else\Gin@nat@height\fi}
\makeatother
% Scale images if necessary, so that they will not overflow the page
% margins by default, and it is still possible to overwrite the defaults
% using explicit options in \includegraphics[width, height, ...]{}
\setkeys{Gin}{width=\maxwidth,height=\maxheight,keepaspectratio}
% Set default figure placement to htbp
\makeatletter
\def\fps@figure{htbp}
\makeatother
\setlength{\emergencystretch}{3em} % prevent overfull lines
\providecommand{\tightlist}{%
  \setlength{\itemsep}{0pt}\setlength{\parskip}{0pt}}
\setcounter{secnumdepth}{-\maxdimen} % remove section numbering
\ifLuaTeX
  \usepackage{selnolig}  % disable illegal ligatures
\fi
\IfFileExists{bookmark.sty}{\usepackage{bookmark}}{\usepackage{hyperref}}
\IfFileExists{xurl.sty}{\usepackage{xurl}}{} % add URL line breaks if available
\urlstyle{same} % disable monospaced font for URLs
\hypersetup{
  pdftitle={Reproducible Research: Project\_1},
  pdfauthor={Mansour\_haghi},
  hidelinks,
  pdfcreator={LaTeX via pandoc}}

\title{Reproducible Research: Project\_1}
\author{Mansour\_haghi}
\date{2022-12-31}

\begin{document}
\maketitle

\hypertarget{introduction}{%
\subsection{Introduction}\label{introduction}}

It is now possible to collect a large amount of data about personal
movement using activity monitoring devices such as a
\href{https://www.fitbit.com/home}{Fitbit},
\href{https://www.nike.com/us/en_us/c/nikeplus-fuelband}{Nike Fuelband},
or \href{https://jawbone.com/up}{Jawbone Up}. These type of devices are
part of the ``quantified self'' movement -- a group of enthusiasts who
take measurements about themselves regularly to improve their health, to
find patterns in their behavior, or because they are tech geeks. But
these data remain under-utilized both because the raw data are hard to
obtain and there is a lack of statistical methods and software for
processing and interpreting the data.

This assignment makes use of data from a personal activity monitoring
device. This device collects data at 5 minute intervals through out the
day. The data consists of two months of data from an anonymous
individual collected during the months of October and November, 2012 and
include the number of steps taken in 5 minute intervals each day.

The data for this assignment can be downloaded from the course web site:

The variables included in this dataset are:

\begin{itemize}
\tightlist
\item
  \textbf{steps}: Number of steps taking in a 5-minute interval (missing
  values are coded as \emph{NA})
\item
  \textbf{date}: The date on which the measurement was taken in
  YYYY-MM-DD format
\item
  \textbf{interval}: Identifier for the 5-minute interval in which
  measurement was taken The dataset is stored in a comma-separated-value
  (CSV) file and there are a total of 17,568 observations in this
  dataset.
\end{itemize}

\hypertarget{required-packages}{%
\subsection{required packages}\label{required-packages}}

\begin{Shaded}
\begin{Highlighting}[]
\CommentTok{\# Loading packages}
\FunctionTok{library}\NormalTok{(ggplot2)}
\FunctionTok{library}\NormalTok{(ggthemes)}
\FunctionTok{library}\NormalTok{(scales)}
\FunctionTok{library}\NormalTok{(lubridate)}
\end{Highlighting}
\end{Shaded}

\hypertarget{loading-and-preprocessing-the-data}{%
\subsection{Loading and preprocessing the
data}\label{loading-and-preprocessing-the-data}}

Downloading and unzipping data.

\begin{Shaded}
\begin{Highlighting}[]
\NormalTok{path }\OtherTok{=} \FunctionTok{getwd}\NormalTok{()}
\FunctionTok{unzip}\NormalTok{(}\StringTok{"repdata\_data\_activity.zip"}\NormalTok{, }\AttributeTok{exdir =}\NormalTok{ path)}
\end{Highlighting}
\end{Shaded}

Reading csv and summary.

\begin{Shaded}
\begin{Highlighting}[]
\NormalTok{activity }\OtherTok{\textless{}{-}} \FunctionTok{read.csv}\NormalTok{(}\StringTok{"activity.csv"}\NormalTok{)}

\NormalTok{activity}\SpecialCharTok{$}\NormalTok{date }\OtherTok{\textless{}{-}} \FunctionTok{as.POSIXct}\NormalTok{(activity}\SpecialCharTok{$}\NormalTok{date, }\StringTok{"\%Y\%m\%d"}\NormalTok{)}

\NormalTok{day }\OtherTok{\textless{}{-}} \FunctionTok{weekdays}\NormalTok{(activity}\SpecialCharTok{$}\NormalTok{date)}

\NormalTok{activity }\OtherTok{\textless{}{-}} \FunctionTok{cbind}\NormalTok{(activity, day)}

\FunctionTok{summary}\NormalTok{(activity)}
\end{Highlighting}
\end{Shaded}

\begin{verbatim}
##      steps             date               interval          day           
##  Min.   :  0.00   Min.   :2012-10-01   Min.   :   0.0   Length:17568      
##  1st Qu.:  0.00   1st Qu.:2012-10-16   1st Qu.: 588.8   Class :character  
##  Median :  0.00   Median :2012-10-31   Median :1177.5   Mode  :character  
##  Mean   : 37.38   Mean   :2012-10-31   Mean   :1177.5                     
##  3rd Qu.: 12.00   3rd Qu.:2012-11-15   3rd Qu.:1766.2                     
##  Max.   :806.00   Max.   :2012-11-30   Max.   :2355.0                     
##  NA's   :2304
\end{verbatim}

\hypertarget{what-is-mean-total-number-of-steps-taken-per-day}{%
\subsection{What is mean total number of steps taken per
day?}\label{what-is-mean-total-number-of-steps-taken-per-day}}

\hypertarget{calculating-total-steps-taken-on-a-day-_-changing-col-_converting-the-data-set-into-a-data-frame-to-be-able-to-use-ggplot2-names_plotting-a-histogram-using-ggplot2}{%
\subsection{Calculating total steps taken on a day \_ Changing col
\_Converting the data set into a data frame to be able to use ggplot2
names\_Plotting a histogram using
ggplot2}\label{calculating-total-steps-taken-on-a-day-_-changing-col-_converting-the-data-set-into-a-data-frame-to-be-able-to-use-ggplot2-names_plotting-a-histogram-using-ggplot2}}

\hypertarget{creating-a-histogram-of-the-total-number-of-steps-taken-each-day.}{%
\subsubsection{Creating a Histogram of the total number of steps taken
each
day.}\label{creating-a-histogram-of-the-total-number-of-steps-taken-each-day.}}

\begin{Shaded}
\begin{Highlighting}[]
\NormalTok{activity\_T\_Steps }\OtherTok{\textless{}{-}} \FunctionTok{with}\NormalTok{(activity, }\FunctionTok{aggregate}\NormalTok{(steps, }\AttributeTok{by =} \FunctionTok{list}\NormalTok{(date), sum, }\AttributeTok{na.rm =} \ConstantTok{TRUE}\NormalTok{))}
\FunctionTok{names}\NormalTok{(activity\_T\_Steps) }\OtherTok{\textless{}{-}} \FunctionTok{c}\NormalTok{(}\StringTok{"Date"}\NormalTok{, }\StringTok{"Steps"}\NormalTok{)}

\NormalTok{total\_Steps\_d }\OtherTok{\textless{}{-}} \FunctionTok{data.frame}\NormalTok{(activity\_T\_Steps)}

\NormalTok{plot1 }\OtherTok{\textless{}{-}} \FunctionTok{ggplot}\NormalTok{(total\_Steps\_d, }\FunctionTok{aes}\NormalTok{(}\AttributeTok{x =}\NormalTok{ Steps)) }\SpecialCharTok{+} 
  \FunctionTok{geom\_histogram}\NormalTok{(}\AttributeTok{breaks =} \FunctionTok{seq}\NormalTok{(}\DecValTok{0}\NormalTok{, }\DecValTok{25000}\NormalTok{, }\AttributeTok{by =} \DecValTok{2500}\NormalTok{), }\AttributeTok{fill =} \StringTok{"\#f383ff"}\NormalTok{, }\AttributeTok{col =} \StringTok{"red"}\NormalTok{) }\SpecialCharTok{+} 
  \FunctionTok{ylim}\NormalTok{(}\DecValTok{0}\NormalTok{, }\DecValTok{30}\NormalTok{) }\SpecialCharTok{+} 
  \FunctionTok{xlab}\NormalTok{(}\StringTok{"Total Steps Taken Per Day"}\NormalTok{) }\SpecialCharTok{+} 
  \FunctionTok{ylab}\NormalTok{(}\StringTok{"Frequency"}\NormalTok{) }\SpecialCharTok{+} 
  \FunctionTok{ggtitle}\NormalTok{(}\StringTok{"Total Number of Steps Taken on a Day"}\NormalTok{) }\SpecialCharTok{+} 
  \FunctionTok{theme\_classic}\NormalTok{(}\AttributeTok{base\_family =} \StringTok{"serif"}\NormalTok{)}

\FunctionTok{print}\NormalTok{(plot1)}
\end{Highlighting}
\end{Shaded}

\includegraphics{figures unnamed-chunk-4-1.png}

\hypertarget{the-mean-and-median-of-steps-taken-per-day}{%
\subsection{The mean and median of steps taken per
day}\label{the-mean-and-median-of-steps-taken-per-day}}

\begin{Shaded}
\begin{Highlighting}[]
\FunctionTok{mean}\NormalTok{(activity\_T\_Steps}\SpecialCharTok{$}\NormalTok{Steps, }\AttributeTok{na.rm =}\NormalTok{ T)}
\end{Highlighting}
\end{Shaded}

\begin{verbatim}
## [1] 9354.23
\end{verbatim}

\begin{Shaded}
\begin{Highlighting}[]
\FunctionTok{median}\NormalTok{(activity\_T\_Steps}\SpecialCharTok{$}\NormalTok{Steps, }\AttributeTok{na.rm =}\NormalTok{ T)}
\end{Highlighting}
\end{Shaded}

\begin{verbatim}
## [1] 10395
\end{verbatim}

\hypertarget{what-is-the-average-daily-activity-pattern}{%
\subsection{What is the average daily activity
pattern?}\label{what-is-the-average-daily-activity-pattern}}

Calculating the average steps taken for each 5-minute interval. \#\#
Time Series plot

\begin{Shaded}
\begin{Highlighting}[]
\NormalTok{av\_Daily\_Activity }\OtherTok{\textless{}{-}} \FunctionTok{aggregate}\NormalTok{(activity}\SpecialCharTok{$}\NormalTok{steps, }\AttributeTok{by =} \FunctionTok{list}\NormalTok{(activity}\SpecialCharTok{$}\NormalTok{interval), }\AttributeTok{FUN =}\NormalTok{ mean, }\AttributeTok{na.rm =} \ConstantTok{TRUE}\NormalTok{)}
\end{Highlighting}
\end{Shaded}

Changing col names and Converting the data set into a dataframe.

\begin{Shaded}
\begin{Highlighting}[]
\FunctionTok{names}\NormalTok{(av\_Daily\_Activity) }\OtherTok{\textless{}{-}} \FunctionTok{c}\NormalTok{(}\StringTok{"Interval"}\NormalTok{, }\StringTok{"Mean"}\NormalTok{)}

\NormalTok{av\_Activity\_d }\OtherTok{\textless{}{-}} \FunctionTok{data.frame}\NormalTok{(av\_Daily\_Activity)}
\end{Highlighting}
\end{Shaded}

we have the valid date-time format we can have a cleaner looking
time-series plot, for the average 24-hour period.

\begin{Shaded}
\begin{Highlighting}[]
\NormalTok{plot2 }\OtherTok{\textless{}{-}} \FunctionTok{ggplot}\NormalTok{(av\_Activity\_d, }\AttributeTok{mapping =} \FunctionTok{aes}\NormalTok{(Interval, Mean)) }\SpecialCharTok{+} 
  \FunctionTok{geom\_line}\NormalTok{(}\AttributeTok{col =} \StringTok{"green"}\NormalTok{) }\SpecialCharTok{+}
  \FunctionTok{xlab}\NormalTok{(}\StringTok{"Interval"}\NormalTok{) }\SpecialCharTok{+} 
  \FunctionTok{ylab}\NormalTok{(}\StringTok{"Average Number of Steps"}\NormalTok{) }\SpecialCharTok{+} 
  \FunctionTok{ggtitle}\NormalTok{(}\StringTok{"Average Number of Steps Per Interval"}\NormalTok{) }\SpecialCharTok{+}
  \FunctionTok{theme\_dark}\NormalTok{(}\AttributeTok{base\_family =} \StringTok{"serif"}\NormalTok{)}
  
\FunctionTok{print}\NormalTok{(plot2)}
\end{Highlighting}
\end{Shaded}

\includegraphics{figures unnamed-chunk-8-1.png}

\hypertarget{which-5-minute-interval-on-average-across-all-the-days-in-the-dataset-contains-the-maximum-number-of-steps}{%
\subsection{Which 5-minute interval, on average across all the days in
the dataset, contains the maximum number of
steps?}\label{which-5-minute-interval-on-average-across-all-the-days-in-the-dataset-contains-the-maximum-number-of-steps}}

\begin{Shaded}
\begin{Highlighting}[]
\NormalTok{av\_Daily\_Activity[}\FunctionTok{which.max}\NormalTok{(av\_Daily\_Activity}\SpecialCharTok{$}\NormalTok{Mean), ]}\SpecialCharTok{$}\NormalTok{Interval}
\end{Highlighting}
\end{Shaded}

\begin{verbatim}
## [1] 835
\end{verbatim}

\hypertarget{imputing-missing-data}{%
\subsection{Imputing missing data}\label{imputing-missing-data}}

The total amount of NA's and the percentage of missing step data.

\begin{Shaded}
\begin{Highlighting}[]
\FunctionTok{sum}\NormalTok{(}\FunctionTok{is.na}\NormalTok{(activity}\SpecialCharTok{$}\NormalTok{steps))}
\end{Highlighting}
\end{Shaded}

\begin{verbatim}
## [1] 2304
\end{verbatim}

Matching the mean of daily activity with the missing values

\begin{Shaded}
\begin{Highlighting}[]
\NormalTok{Imputed\_Steps }\OtherTok{\textless{}{-}}\NormalTok{av\_Daily\_Activity}\SpecialCharTok{$}\NormalTok{Mean[}\FunctionTok{match}\NormalTok{(activity}\SpecialCharTok{$}\NormalTok{interval, av\_Daily\_Activity}\SpecialCharTok{$}\NormalTok{Interval)]}
\end{Highlighting}
\end{Shaded}

Transforming steps in activity if they were missing values with the
filled values from above. Forming the new dataset with the imputed
missing values. Changing col names

\begin{Shaded}
\begin{Highlighting}[]
\NormalTok{activity\_Imputed }\OtherTok{\textless{}{-}} \FunctionTok{transform}\NormalTok{(activity, }
                             \AttributeTok{steps =} \FunctionTok{ifelse}\NormalTok{(}\FunctionTok{is.na}\NormalTok{(activity}\SpecialCharTok{$}\NormalTok{steps), }\AttributeTok{yes =}\NormalTok{ Imputed\_Steps, }\AttributeTok{no =}\NormalTok{ activity}\SpecialCharTok{$}\NormalTok{steps))}


\NormalTok{totalactivity\_Imputed }\OtherTok{\textless{}{-}} \FunctionTok{aggregate}\NormalTok{(steps }\SpecialCharTok{\textasciitilde{}}\NormalTok{ date, activity\_Imputed, sum)}


\FunctionTok{names}\NormalTok{(totalactivity\_Imputed) }\OtherTok{\textless{}{-}} \FunctionTok{c}\NormalTok{(}\StringTok{"date"}\NormalTok{, }\StringTok{"dailySteps"}\NormalTok{)}

\FunctionTok{sum}\NormalTok{(}\FunctionTok{is.na}\NormalTok{(totalactivity\_Imputed}\SpecialCharTok{$}\NormalTok{dailySteps))}
\end{Highlighting}
\end{Shaded}

\begin{verbatim}
## [1] 0
\end{verbatim}

\hypertarget{histogram-of-the-total-number-of-steps-taken-each-day-after-missing-values-were-imputed}{%
\subsection{Histogram of the total number of steps taken each day after
missing values were
imputed}\label{histogram-of-the-total-number-of-steps-taken-each-day-after-missing-values-were-imputed}}

\hypertarget{converting-the-data-set-into-a-data-frame-to-be-able-to-use-ggplot2}{%
\subsection{Converting the data set into a data frame to be able to use
ggplot2}\label{converting-the-data-set-into-a-data-frame-to-be-able-to-use-ggplot2}}

\hypertarget{plotting-a-histogram-using-ggplot2}{%
\subsection{Plotting a histogram using
ggplot2}\label{plotting-a-histogram-using-ggplot2}}

\begin{Shaded}
\begin{Highlighting}[]
\NormalTok{totalImputed\_Stepsdf }\OtherTok{\textless{}{-}} \FunctionTok{data.frame}\NormalTok{(totalactivity\_Imputed)}

\NormalTok{p }\OtherTok{\textless{}{-}} \FunctionTok{ggplot}\NormalTok{(totalImputed\_Stepsdf, }\FunctionTok{aes}\NormalTok{(}\AttributeTok{x =}\NormalTok{ dailySteps)) }\SpecialCharTok{+} 
  \FunctionTok{geom\_histogram}\NormalTok{(}\AttributeTok{breaks =} \FunctionTok{seq}\NormalTok{(}\DecValTok{0}\NormalTok{, }\DecValTok{25000}\NormalTok{, }\AttributeTok{by =} \DecValTok{2500}\NormalTok{), }\AttributeTok{fill =} \StringTok{"\#f383ff"}\NormalTok{, }\AttributeTok{col =} \StringTok{"black"}\NormalTok{) }\SpecialCharTok{+} 
  \FunctionTok{ylim}\NormalTok{(}\DecValTok{0}\NormalTok{, }\DecValTok{30}\NormalTok{) }\SpecialCharTok{+} 
  \FunctionTok{xlab}\NormalTok{(}\StringTok{"Total Steps Taken Per Day"}\NormalTok{) }\SpecialCharTok{+} 
  \FunctionTok{ylab}\NormalTok{(}\StringTok{"Frequency"}\NormalTok{) }\SpecialCharTok{+} 
  \FunctionTok{ggtitle}\NormalTok{(}\StringTok{"Total Number of Steps Taken on a Day"}\NormalTok{) }\SpecialCharTok{+} 
  \FunctionTok{theme\_classic}\NormalTok{(}\AttributeTok{base\_family =} \StringTok{"serif"}\NormalTok{)}

\FunctionTok{print}\NormalTok{(p)}
\end{Highlighting}
\end{Shaded}

\includegraphics{figures unnamed-chunk-13-1.png} The mean of the total
number of steps taken per day is:

\begin{Shaded}
\begin{Highlighting}[]
\FunctionTok{mean}\NormalTok{(totalactivity\_Imputed}\SpecialCharTok{$}\NormalTok{dailySteps)}
\end{Highlighting}
\end{Shaded}

\begin{verbatim}
## [1] 10766.19
\end{verbatim}

The median of the total number of steps taken per day is:

\begin{Shaded}
\begin{Highlighting}[]
\FunctionTok{median}\NormalTok{(totalactivity\_Imputed}\SpecialCharTok{$}\NormalTok{dailySteps)}
\end{Highlighting}
\end{Shaded}

\begin{verbatim}
## [1] 10766.19
\end{verbatim}

\hypertarget{are-there-differences-in-activity-patterns-between-weekdays-and-weekends}{%
\subsection{Are there differences in activity patterns between weekdays
and
weekends?}\label{are-there-differences-in-activity-patterns-between-weekdays-and-weekends}}

\hypertarget{panel-plot-comparing-the-average-number-of-steps-taken-per-5-minute-intervals-for-weekdays-and-weekends}{%
\subsection{Panel plot comparing the average number of steps taken per
5-minute intervals for Weekdays and
Weekends}\label{panel-plot-comparing-the-average-number-of-steps-taken-per-5-minute-intervals-for-weekdays-and-weekends}}

Coverting `interval' column data to a vailid date-time format.

\begin{Shaded}
\begin{Highlighting}[]
\NormalTok{activity}\SpecialCharTok{$}\NormalTok{date }\OtherTok{\textless{}{-}} \FunctionTok{as.Date}\NormalTok{(}\FunctionTok{strptime}\NormalTok{(activity}\SpecialCharTok{$}\NormalTok{date, }\AttributeTok{format=}\StringTok{"\%Y{-}\%m{-}\%d"}\NormalTok{))}

\NormalTok{activity}\SpecialCharTok{$}\NormalTok{dayType }\OtherTok{\textless{}{-}} \FunctionTok{sapply}\NormalTok{(activity}\SpecialCharTok{$}\NormalTok{date, }\ControlFlowTok{function}\NormalTok{(x) \{}
  \ControlFlowTok{if}\NormalTok{(}\FunctionTok{weekdays}\NormalTok{(x) }\SpecialCharTok{==} \StringTok{"شنبه"} \SpecialCharTok{|} \FunctionTok{weekdays}\NormalTok{(x) }\SpecialCharTok{==} \StringTok{"یکشنبه"}\NormalTok{)}
\NormalTok{  \{AA }\OtherTok{\textless{}{-}} \StringTok{"Weekend"}\NormalTok{\}}
  \ControlFlowTok{else}\NormalTok{ \{AA }\OtherTok{\textless{}{-}} \StringTok{"Weekday"}\NormalTok{\}}
\NormalTok{  AA}
\NormalTok{\})}
\end{Highlighting}
\end{Shaded}

\hypertarget{creating-the-data-set-that-will-be-plotted-and-plotting-using-ggplot2}{%
\subsection{Creating the data set that will be plotted and Plotting
using
ggplot2}\label{creating-the-data-set-that-will-be-plotted-and-plotting-using-ggplot2}}

Now that we have the valid date time format we can have a cleaner
looking Time-Seriesplot, for comparison of the average 24-hour period on
weekdays and weekends.

\begin{Shaded}
\begin{Highlighting}[]
\NormalTok{activity\_inDay }\OtherTok{\textless{}{-}}  \FunctionTok{aggregate}\NormalTok{(steps }\SpecialCharTok{\textasciitilde{}}\NormalTok{ interval }\SpecialCharTok{+}\NormalTok{ dayType, activity, mean, }\AttributeTok{na.rm =} \ConstantTok{TRUE}\NormalTok{)}


\NormalTok{plot3 }\OtherTok{\textless{}{-}}  \FunctionTok{ggplot}\NormalTok{(activity\_inDay, }\FunctionTok{aes}\NormalTok{(}\AttributeTok{x =}\NormalTok{ interval , }\AttributeTok{y =}\NormalTok{ steps, }\AttributeTok{color =}\NormalTok{ dayType)) }\SpecialCharTok{+} 
  \FunctionTok{geom\_line}\NormalTok{() }\SpecialCharTok{+} \FunctionTok{ggtitle}\NormalTok{(}\StringTok{"average daily steps by say type"}\NormalTok{) }\SpecialCharTok{+} 
  \FunctionTok{xlab}\NormalTok{(}\StringTok{"interval"}\NormalTok{) }\SpecialCharTok{+} 
  \FunctionTok{ylab}\NormalTok{(}\StringTok{"average number of Steps"}\NormalTok{) }\SpecialCharTok{+}
  \FunctionTok{facet\_wrap}\NormalTok{(}\SpecialCharTok{\textasciitilde{}}\NormalTok{dayType, }\AttributeTok{ncol =} \DecValTok{1}\NormalTok{, }\AttributeTok{nrow=}\DecValTok{2}\NormalTok{) }\SpecialCharTok{+}
  \FunctionTok{scale\_color\_discrete}\NormalTok{(}\AttributeTok{name =} \StringTok{"day type"}\NormalTok{) }\SpecialCharTok{+}
  \FunctionTok{theme\_economist}\NormalTok{(}\AttributeTok{base\_family =} \StringTok{"serif"}\NormalTok{)}

\FunctionTok{print}\NormalTok{(plot3) }
\end{Highlighting}
\end{Shaded}

\includegraphics{figures unnamed-chunk-17-1.png}

\end{document}
